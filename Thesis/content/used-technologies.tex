\chapter{Felhasznált eszközök}

\section{Altium Designer}

Az Altium Designer egy teljes körű elektronikai tervező szoftvercsomag (EDA), amelyet nyomtatott áramköri lapok (PCB-k) tervezésére alkalmaznak. Egységes környezetben kombinálja a kapcsolás rajz és a PCB tervezését, ezzel lehetővé téve a tervezőknek a grafikus áramköri tervek gyors és hatékony létrehozását. A szoftver támogatja a hierarchikus tervezést és a kapcsolási rajzok valós idejű ellenőrzését. A PCB tervezés során előnyös funkciói közé tartozik az interaktív huzalozás, a 3D-s PCB megjelenítés és a különböző tervezési szabályok ellenőrzése.

Az Altium Designer integrált megközelítése lehetővé teszi a tervezők számára, hogy egyetlen platformon belül hozzák létre és optimalizálják elektronikai terveiket. A szoftver nemcsak a tervezési folyamatot egyszerűsíti, hanem magában foglalja a szimulációs és elemző eszközöket is, amelyek segítik a tervezőket az elektronikai rendszerek teljesítményének és megbízhatóságának előzetes értékelésében.

\section{Xilinx ISE}

A projektben használt Spartan-6-os FPGA-t a Xilinx új fejlesztő környezete, a Vivado már nem támogatja, ezért a fejlesztéseket a régebbi Xilinx ISE-ben kellett elkészítenem. A Xilinx ISE (Integrated Softver Environment) a Xilinx Inc. által kifejlesztett, széles körben használt szoftvercsomag. Átfogó eszközkészletet biztosít a digitális logikai áramkörök tervezéséhez, teszteléséhez és megvalósításához a Xilinx Field-Programmable Gate Array (FPGA) és Complex Programmable Logic Device (CPLD) eszközökkel.

A Xilinx ISE teljes körű megoldást kínál a digitális tervezéshez, beleértve a HDL (Hardware Description Language) tervezést, a szimulációt, a szintézist, az implementációt és az eszközprogramozást. Támogatja a különböző tervezési beviteli módszereket, beleértve a Xilinx Schematic Editor segítségével történő sematikus rögzítést és a HDL-alapú tervezést olyan nyelvekkel, mint a VHDL (VHSIC Hardware Description Language) és a Verilog.


%----------------------------------------------------------------------------
\chapter{\bevezetes}
%----------------------------------------------------------------------------

Az 1983-ban megjelent Nintendo Entertainment System (NES) 8 bites videojáték konzol a
maga korában igen népszerű volt. A hardverének kialakítása több későbbi, modernebb
videojáték konzolra volt hatással, valamint számos kiemelkedő játékprogram erre a konzolra
készült el először.

A diplomatervem során a NES viszonylag egyszerű, jól átgondolt hardverét valósítottam meg egy olcsóbb, viszonylag kevesebb erőforrással rendelkező FPGA eszközzel. Ennek a megvalósításnak több előnye is van, például ki lehet használni a modern megjelenítő interfészeket (VGA, DVI, HDMI, stb.), valamint az eredeti játékkazetta helyett alkalmazni lehet modern adattároló eszközöket is (SD kártya). Természetesen ezen „továbbfejlesztések” mellett az egyedi változatnak képesnek kell lennie futtatni az eredeti játékprogramokat.

A következőkben a projekt során elkészült egyedi, FPGA alapú NES megvalósítás hardver és szoftveres komponenseinek bemutatását olvashatjuk. Először is az eredeti NES működését, majd a hardver tovább fejlesztését ismerhetjük meg. Ezt követően pedig a FPGA NES továbbfejlesztett alaplapját, és az FPGA-ban elkészült komponensek működését és tesztelését láthatjuk.

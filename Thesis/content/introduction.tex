%----------------------------------------------------------------------------
\chapter{\bevezetes}
%----------------------------------------------------------------------------

Az 1983-ban megjelent Nintendo Entertainment System (NES) 8 bites videojáték konzol a
maga korában igen népszerű volt. A hardverének kialakítása több későbbi, modernebb
videojáték konzolra volt hatással, valamint számos kiemelkedő játékprogram erre a konzolra
készült el először.

A NES viszonylag egyszerű, jól átgondolt hardvere lehetővé teszi annak az olcsóbb, kevesebb
erőforrással rendelkező FPGA eszközökkel történő megvalósítását. Egy ilyen
megvalósításnak több előnye is van, például ki lehet használni a modern megjelenítő
interfészeket (VGA, DVI, HDMI, stb.), valamint az eredeti játékkazetta helyett alkalmazni
lehet modern adattároló eszközöket is (SD kártya). Természetesen ezen „továbbfejlesztések”
mellett az egyedi változat képes futtatni az eredeti játékprogramokat.

A feladat célja egy egyedi, FPGA alapú NES megvalósítás hardver és szoftver
komponenseinek elkészítése. 


%--------------------------------------------------------------------------------------
% Page layout setup
%--------------------------------------------------------------------------------------
% we need to redefine the pagestyle plain
% another possibility is to use the body of this command without \fancypagestyle
% and use \pagestyle{fancy} but in that case the special pages
% (like the ToC, the References, and the Chapter pages)remain in plane style

\pagestyle{plain}
\marginsize{35mm}{25mm}{15mm}{15mm}

\setcounter{tocdepth}{3}
%\sectionfont{\large\upshape\bfseries}
\setcounter{secnumdepth}{3}

\usepackage[labelsep=colon]{caption}

\sloppy % Margón túllógó sorok tiltása.
\widowpenalty=10000 \clubpenalty=10000 %A fattyú- és árvasorok elkerülése
\def\hyph{-\penalty0\hskip0pt\relax} % Kötőjeles szavak elválasztásának engedélyezése


%--------------------------------------------------------------------------------------
% Setup hyperref package
%--------------------------------------------------------------------------------------
\hypersetup{
    % bookmarks=true,            % show bookmarks bar?
    unicode=true,              % non-Latin characters in Acrobat's bookmarks
    pdftitle={\vikcim},        % title
    pdfauthor={\szerzoMeta},    % author
    pdfsubject={\vikdoktipus}, % subject of the document
    pdfcreator={\szerzoMeta},   % creator of the document
    pdfproducer={},    % producer of the document
    pdfkeywords={},    % list of keywords (separate then by comma)
    pdfnewwindow=true,         % links in new window
    colorlinks=true,           % false: boxed links; true: colored links
    linkcolor=black,           % color of internal links
    citecolor=black,           % color of links to bibliography
    filecolor=black,           % color of file links
    urlcolor=black             % color of external links
}


%--------------------------------------------------------------------------------------
% Set up listings
%--------------------------------------------------------------------------------------
\definecolor{lightgray}{rgb}{0.95,0.95,0.95}
\lstset{
	basicstyle=\scriptsize\ttfamily, % print whole listing small
	keywordstyle=\color{black}\bfseries, % bold black keywords
	identifierstyle=, % nothing happens
	% default behavior: comments in italic, to change use
	% commentstyle=\color{green}, % for e.g. green comments
	stringstyle=\scriptsize,
	showstringspaces=false, % no special string spaces
	aboveskip=3pt,
	belowskip=3pt,
	backgroundcolor=\color{lightgray},
	columns=flexible,
	keepspaces=true,
	escapeinside={(*@}{@*)},
	captionpos=b,
	breaklines=true,
	frame=single,
	float=!ht,
	tabsize=2,
	literate=*
		{á}{{\'a}}1	{é}{{\'e}}1	{í}{{\'i}}1	{ó}{{\'o}}1	{ö}{{\"o}}1	{ő}{{\H{o}}}1	{ú}{{\'u}}1	{ü}{{\"u}}1	{ű}{{\H{u}}}1
		{Á}{{\'A}}1	{É}{{\'E}}1	{Í}{{\'I}}1	{Ó}{{\'O}}1	{Ö}{{\"O}}1	{Ő}{{\H{O}}}1	{Ú}{{\'U}}1	{Ü}{{\"U}}1	{Ű}{{\H{U}}}1
}

%--------------------------------------------------------------------------------------
% Verilog Code Style
%--------------------------------------------------------------------------------------
\definecolor{verilogcommentcolor}{RGB}{104,180,104}
\definecolor{verilogkeywordcolor}{RGB}{49,49,255}
\definecolor{verilogsystemcolor}{RGB}{128,0,255}
\definecolor{verilognumbercolor}{RGB}{255,143,102}
\definecolor{verilogstringcolor}{RGB}{160,160,160}
\definecolor{verilogdefinecolor}{RGB}{128,64,0}
\definecolor{verilogoperatorcolor}{RGB}{0,0,128}

% Verilog style
\lstdefinestyle{prettyverilog}{
	language           = Verilog,
	commentstyle       = \color{verilogcommentcolor},
	alsoletter         = \$'0123456789\`,
	literate           = *{+}{{\verilogColorOperator{+}}}{1}%
	{-}{{\verilogColorOperator{-}}}{1}%
	{@}{{\verilogColorOperator{@}}}{1}%
	{;}{{\verilogColorOperator{;}}}{1}%
	{*}{{\verilogColorOperator{*}}}{1}%
	{?}{{\verilogColorOperator{?}}}{1}%
	{:}{{\verilogColorOperator{:}}}{1}%
	{<}{{\verilogColorOperator{<}}}{1}%
	{>}{{\verilogColorOperator{>}}}{1}%
	{=}{{\verilogColorOperator{=}}}{1}%
	{!}{{\verilogColorOperator{!}}}{1}%
	{^}{{\verilogColorOperator{$\land$}}}{1}%
	{|}{{\verilogColorOperator{|}}}{1}%
	{=}{{\verilogColorOperator{=}}}{1}%
	{[}{{\verilogColorOperator{[}}}{1}%
	{]}{{\verilogColorOperator{]}}}{1}%
	{(}{{\verilogColorOperator{(}}}{1}%
	{)}{{\verilogColorOperator{)}}}{1}%
	{,}{{\verilogColorOperator{,}}}{1}%
	{.}{{\verilogColorOperator{.}}}{1}%
	{~}{{\verilogColorOperator{$\sim$}}}{1}%
	{\%}{{\verilogColorOperator{\%}}}{1}%
	{\&}{{\verilogColorOperator{\&}}}{1}%
	{\#}{{\verilogColorOperator{\#}}}{1}%
	{\ /\ }{{\verilogColorOperator{\ /\ }}}{3}%
	{\ _}{\ \_}{2}%
	,
	morestring         = [s][\color{verilogstringcolor}]{"}{"},%
	identifierstyle    = \color{black},
	vlogdefinestyle    = \color{verilogdefinecolor},
	vlogconstantstyle  = \color{verilognumbercolor},
	vlogsystemstyle    = \color{verilogsystemcolor},
	basicstyle         = \scriptsize\fontencoding{T1}\ttfamily,
	keywordstyle       = \bfseries\color{verilogkeywordcolor},
	numbers            = left,
	numbersep          = 10pt,
	tabsize            = 4,
	escapeinside       = {/*!}{!*/},
	upquote            = true,
	sensitive          = true,
	showstringspaces   = false, %without this there will be a symbol in the places where there is a space
	frame              = single
}


% This is shamelessly stolen and modified from:
% https://github.com/jubobs/sclang-prettifier/blob/master/sclang-prettifier.dtx
\makeatletter

% Language name
\newcommand\language@verilog{Verilog}
\expandafter\lst@NormedDef\expandafter\languageNormedDefd@verilog%
\expandafter{\language@verilog}

% save definition of single quote for testing
\lst@SaveOutputDef{`'}\quotesngl@verilog
\lst@SaveOutputDef{``}\backtick@verilog
\lst@SaveOutputDef{`\$}\dollar@verilog

% Extract first character token in sequence and store in macro 
% firstchar@verilog, per http://tex.stackexchange.com/a/159267/21891
\newcommand\getfirstchar@verilog{}
\newcommand\getfirstchar@@verilog{}
\newcommand\firstchar@verilog{}
\def\getfirstchar@verilog#1{\getfirstchar@@verilog#1\relax}
\def\getfirstchar@@verilog#1#2\relax{\def\firstchar@verilog{#1}}

% Initially empty hook for lst
\newcommand\addedToOutput@verilog{}
\lst@AddToHook{Output}{\addedToOutput@verilog}

% The style used for constants as set in lstdefinestyle
\newcommand\constantstyle@verilog{}
\lst@Key{vlogconstantstyle}\relax%
{\def\constantstyle@verilog{#1}}

% The style used for defines as set in lstdefinestyle
\newcommand\definestyle@verilog{}
\lst@Key{vlogdefinestyle}\relax%
{\def\definestyle@verilog{#1}}

% The style used for defines as set in lstdefinestyle
\newcommand\systemstyle@verilog{}
\lst@Key{vlogsystemstyle}\relax%
{\def\systemstyle@verilog{#1}}

% Counter used to check current character is a digit
\newcount\currentchar@verilog

% Processing macro
\newcommand\@ddedToOutput@verilog
{%
	% If we're in \lstpkg{}' processing mode...
	\ifnum\lst@mode=\lst@Pmode%
	% Save the first token in the current identifier to \@getfirstchar
	\expandafter\getfirstchar@verilog\expandafter{\the\lst@token}%
	% Check if the token is a backtick
	\expandafter\ifx\firstchar@verilog\backtick@verilog
	% If so, then this starts a define
	\let\lst@thestyle\definestyle@verilog%
	\else
	% Check if the token is a dollar
	\expandafter\ifx\firstchar@verilog\dollar@verilog
	% If so, then this starts a system command
	\let\lst@thestyle\systemstyle@verilog%
	\else
	% Check if the token starts with a single quote
	\expandafter\ifx\firstchar@verilog\quotesngl@verilog
	% If so, then this starts a constant without length
	\let\lst@thestyle\constantstyle@verilog%
	\else
	\currentchar@verilog=48
	\loop
	\expandafter\ifnum%
	\expandafter`\firstchar@verilog=\currentchar@verilog%
	\let\lst@thestyle\constantstyle@verilog%
	\let\iterate\relax%
	\fi
	\advance\currentchar@verilog by \@ne%
	\unless\ifnum\currentchar@verilog>57%
	\repeat%
	\fi
	\fi
	\fi
	% ...but override by keyword style if a keyword is detected!
	%\lsthk@DetectKeywords% 
	\fi
}

% Add processing macro only if verilog
\lst@AddToHook{PreInit}{%
	\ifx\lst@language\languageNormedDefd@verilog%
	\let\addedToOutput@verilog\@ddedToOutput@verilog%
	\fi
}

% Colour operators in literate
\newcommand{\verilogColorOperator}[1]
{%
	\ifnum\lst@mode=\lst@Pmode\relax%
	{\bfseries\textcolor{verilogoperatorcolor}{#1}}%
	\else
	#1%
	\fi
}

\makeatother
%--------------------------------------------------------------------------------------
% End Verilog Code Style
%--------------------------------------------------------------------------------------

%--------------------------------------------------------------------------------------
% Set up theorem-like environments
%--------------------------------------------------------------------------------------
% Using ntheorem package -- see http://www.math.washington.edu/tex-archive/macros/latex/contrib/ntheorem/ntheorem.pdf

\theoremstyle{plain}
\theoremseparator{.}
\newtheorem{example}{\pelda}

\theoremseparator{.}
%\theoremprework{\bigskip\hrule\medskip}
%\theorempostwork{\hrule\bigskip}
\theorembodyfont{\upshape}
\theoremsymbol{{\large \ensuremath{\centerdot}}}
\newtheorem{definition}{\definicio}

\theoremseparator{.}
%\theoremprework{\bigskip\hrule\medskip}
%\theorempostwork{\hrule\bigskip}
\newtheorem{theorem}{\tetel}


%--------------------------------------------------------------------------------------
% Some new commands and declarations
%--------------------------------------------------------------------------------------
\newcommand{\code}[1]{{\upshape\ttfamily\scriptsize\indent #1}}
\newcommand{\doi}[1]{DOI: \href{http://dx.doi.org/\detokenize{#1}}{\raggedright{\texttt{\detokenize{#1}}}}} % A hivatkozások közt így könnyebb DOI-t megadni.

\DeclareMathOperator*{\argmax}{arg\,max}
%\DeclareMathOperator*[1]{\floor}{arg\,max}
\DeclareMathOperator{\sign}{sgn}
\DeclareMathOperator{\rot}{rot}


%--------------------------------------------------------------------------------------
% Setup captions
%--------------------------------------------------------------------------------------
\captionsetup[figure]{
	width=.75\textwidth,
	aboveskip=10pt}

\renewcommand{\captionlabelfont}{\bf}
%\renewcommand{\captionfont}{\footnotesize\it}

%--------------------------------------------------------------------------------------
% Hyphenation exceptions
%--------------------------------------------------------------------------------------
\hyphenation{Shakes-peare Mar-seilles ár-víz-tű-rő tü-kör-fú-ró-gép}

%--------------------------------------------------------------------------------------
% Table float box with bottom caption, box width adjusted to content
%--------------------------------------------------------------------------------------



\author{\vikszerzo}
\title{\viktitle}